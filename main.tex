\documentclass{article}
\usepackage{isotope}
\usepackage{amsmath}

\title{NE 770 Homework 1}
\author{William Dawn}

\begin{document}

\maketitle

\begin{enumerate}
  \item Evaluate the integral
    \begin{equation}
        \int_0^{\pi} \theta \sin(\theta) \; d\theta
    \end{equation}
    by Monte Carlo. Report and plot the mean, the variance, and uncertainty in
    the mean as a function of number of samples. Compare to the analytic 
    solution.
  \item A common expression for the fission neutron spectrum is 
    \begin{equation}
      \chi(E) = 0.453 \exp(-E/0.965) \sinh\left(\sqrt{2.29E}\right)
    \end{equation}
    where $E \in [0,\infty)$ is in MeV. Note $\chi(E)$ is a PDF on the
    interval. The average number of neutrons emitted in fussion from 
    \isotope[235]{U} as a function of energy can be taken to be
    \begin{equation}
      \nu(E) = 
      \begin{cases}
        2.42 + 0.066 E & E \le 1 \\
        2.349 + 0.15 E & E >   1
      \end{cases}
    \end{equation}
    \begin{enumerate}
      \item Show numerically that the standard rejection scheme generates
        energies distributed according to the fission neutron spectrum. Verify
        numerically the efficency of the rejection scheme.
      \item Verigy numerically that increasing the $h_{max}$ value for the
        rejection scheme still generates energies distributed according to the
        fission neutron spectrum while only affecting the efficiency of the
        scheme.
      \item Develop a two region partioned rejection scheme for sampling the
        fisiosn neutron spectrum. Determine the maximum efficiency that can be
        obtained with the two region scheme. Show that the partitioned scheme
        generates energies according to the fission neutron spectrum. Verify
        numerically the efficiency of the partitioned scheme.
      \item 
        \label{part:standard}
        Use the standard rejection scheme to determine both the average
        number of neutrons per fision and the average fissions per neutron
        energy. Compare to the ``analytic'' result.
      \item Repeat part \ref{part:standard} using hte maximum efficiency scheme.
    \end{enumerate}
\end{enumerate}

\end{document}
